%%%%%%%%%%%%%%%%%%%%%%%%%%%%%%%%%%%%%%%%%
% Classicthesis-Styled CV
% LaTeX Template
% Version 1.0 (22/2/13)
%
% This template has been downloaded from:
% http://www.LaTeXTemplates.com
%
% Original author:
% Alessandro Plasmati
%
% License:
% CC BY-NC-SA 3.0 (http://creativecommons.org/licenses/by-nc-sa/3.0/)
%
%%%%%%%%%%%%%%%%%%%%%%%%%%%%%%%%%%%%%%%%%

%------------------------------------------------------------------------------
%	PACKAGES AND OTHER DOCUMENT CONFIGURATIONS
%------------------------------------------------------------------------------

\documentclass[a4paper]{scrartcl}

\reversemarginpar % Move the margin to the left of the page 

\newcommand{\MarginText}[1]{\marginpar{\raggedleft\itshape\small#1}} % New command defining the margin text style

\usepackage[utf8]{inputenc}
\usepackage[nochapters]{classicthesis} % Use the classicthesis style for the style of the document
\usepackage[LabelsAligned]{currvita} % Use the currvita style for the layout of the document

\renewcommand{\cvheadingfont}{\LARGE\color{Maroon}} % Font color of your name at the top

\usepackage{hyperref} % Required for adding links	and customizing them
\hypersetup{colorlinks, breaklinks, urlcolor=Maroon, linkcolor=Maroon} % Set link colors

\newlength{\datebox}\settowidth{\datebox}{Spring 2013} % Set the width of the date box in each block

\newcommand{\NewEntry}[3]{\noindent\hangindent=2em\hangafter=0 \parbox{\datebox}{\small \textit{#1}}\hspace{1.5em} #2 #3 % Define a command for each new block - change spacing and font sizes here: #1 is the left margin, #2 is the italic date field and #3 is the position/employer/location field
\vspace{0.5em}} % Add some white space after each new entry

\newcommand{\Description}[1]{\hangindent=2em\hangafter=0\noindent\raggedright\footnotesize{#1}\par\normalsize\vspace{1em}} % Define a command for descriptions of each entry - change spacing and font sizes here

%\usepackage[right=2in]{geometry}
\setlength{\textwidth}{5.1in}

 \date{}

%------------------------------------------------------------------------------

\begin{document}

\thispagestyle{empty} % Stop the page count at the bottom of the first page

%------------------------------------------------------------------------------
%	NAME AND CONTACT INFORMATION SECTION
%------------------------------------------------------------------------------

\begin{cv}{\spacedallcaps{Tuukka Turto}}\vspace{1.5em} 
% Your name

\noindent\spacedlowsmallcaps{Henkil\"{o}tiedot}\vspace{0.5em} 
% Personal information heading

\NewEntry{}{\textit{Syntynyt Lahdessa,}}{27. helmikuuta 1979} 
% Birthplace and date

\NewEntry{email}{\href{mailto:tuukka.turto@oktaeder.net}
{tuukka.turto@oktaeder.net}} 
% Email address

\NewEntry{web}{\href{https://github.com/tuturto/}{https://github.com/tuturto/}}
% Personal website

\NewEntry{ }{\href{http://www.linkedin.com/in/tuukkaturto/}
{http://www.linkedin.com/in/tuukkaturto/}}
% Personal website

\NewEntry{puhelin}{+358 (0)44 556 5984\ \ }
% Phone number(s)

\vspace{1em} 
% Extra white space between the personal information section and goal
\vspace{2em}
\vspace{4em}

%\noindent\spacedlowsmallcaps{Goal}\vspace{1em} 
% Goal heading, could be used for a quotation or short profile instead

%\Description{Gain fundamental experience in my area of interest and
%expertise.}\vspace{2em} % Goal text

%------------------------------------------------------------------------------
%	WORK EXPERIENCE
%------------------------------------------------------------------------------

\noindent\spacedlowsmallcaps{Ty\"{o}kokemnus}\vspace{1em}

\NewEntry{joulu 2013--}{Senior Software Developer, \textsc{Tieto}}

\Description{\MarginText{Tieto Healthcare \& Welfare}L\"{a}\"{a}ketieteen kuvantamisj\"{a}rjestelmien suunnittelu-,
kehitys- ja testausteht\"{a}v\"{a}t.}

\NewEntry{2010--marras 2013}{Systems Architect, \textsc{Digia}}

% \Description{\MarginText{Digia}Specification, design, development and testing of new features for a business critical financial system written with .Net; evaluating new tools and technologies and making proof of concept demos; architectual responsibility for trade and order management, including integration to external systems; working as a scrum master.}

\Description{\MarginText{Digia}Asiakkaille kriittisen j\"{a}rjestelm\"{a}n
m\"{a}\"{a}rittely-,
suunnittelu-, kehitys- ja testausteht\"{a}v\"{a}t .Net alustalla; uusien ty\"{o}kalujen ja
teknologioiden arviointi ja demoaminen; vastuu kauppa- ja
toimeksiantoj\"{a}rjestelm\"{a}st\"{a}, mukaanlukien integraatio ulkoisiin j\"{a}rjestelmiin;
scrum masterina toimiminen.}

%------------------------------------------------

\NewEntry{hein\"{a}--loka 2010}{Senior Software Developer, \textsc{Digia}}

%\Description{\MarginText{Digia}Specification, design, development and testing of new features for a business critical financial system written with .Net; supporting members of the team by providing them technical and business domain knowledge.}

\Description{\MarginText{Digia}Asiakkaille kriittisen j\"{a}rjestelm\"{a}n m\"{a}\"{a}rittely-,
suunnittelu-, kehitys- ja testausteht\"{a}v\"{a}t .Net alustalla; tiimin j\"{a}senten
tukeminen teknisiss\"{a} ja alakohtaisissa ongelmissa.}

%------------------------------------------------

\NewEntry{2009--2010}{Software Designer, \textsc{Digia}}

%\Description{\MarginText{Digia}Error hunting and fixing, developing new features with .Net for business critical financial systems.}

\Description{\MarginText{Digia}Ongelmien etsiminen ja korjaus, uusien
ominaisuuksien kehitt\"{a}minen .Net alustalle rakennettuun j\"{a}rjestelm\"{a}\"{a}n.}

%------------------------------------------------

\NewEntry{2005--2009}{Design Engineer, SW, \textsc{Nokia}}

%\Description{\MarginText{Nokia}Worked as a tools specialist in Build\&Release team. Daily tasks included customer support, customer requirement collection and analysis, coding (Python, Java, Perl), configuration management and build manager work.}

\Description{\MarginText{Nokia}Ty\"{o}kaluspesialisti Build\&Release tiimiss\"{a};
tekninen tuki tiimille; vaatimuksien ker\"{a}\"{a}minen ja analysointi; ohjelmointi
(Python, Java, Perl); konfiguraatiohallinta ja build managerin teht\"{a}v\"{a}t.}

%------------------------------------------------

\NewEntry{2002--2005}{Software Designer, \textsc{Samstock}}

%\Description{\MarginText{Samstock}Worked as a software designer, tasks included designing, implementing and testing n-tier software with VB.Net, database reverse engineering and implementation (Oracle and Microsoft SQL Server).}

\Description{\MarginText{Samstock}Yksitt\"{a}isten komponenttien suunnittelu,
toteutus ja testaus n-tier j\"{a}rjestelm\"{a}ss\"{a} VB.Net ohjelmointikielell\"{a};
tietokannan taaksep\"{a}inmallinnus (Oracle ja Microsoft SQL Server).}

%------------------------------------------------

\vspace{1em} % Extra space between major sections

%------------------------------------------------------------------------------
%	EDUCATION
%------------------------------------------------------------------------------

\spacedlowsmallcaps{Koulutus}\vspace{1em}

%------------------------------------------------

\NewEntry{2011--2014 (arvioitu)}{Jyv\"{a}skyl\"{a}n ammattikorkeakoulu}

\Description{\MarginText{Master of Engineering}Tekniikka ja liikenne\newline 
Lopputy\"{o}: \textit{Automated Testing Performed by Developers}\newline
Kuvaus: Toimintatutkimus kehitt\"{a}jien suorittaman automaattisen testauksen
parantamiseksi. Tuloksena automaattinen testaus otettiin aktiiviseen
k\"{a}ytt\"{o}\"{o}n ja tasoerot kehitt\"{a}jien v\"{a}lill\"{a} kapenivat.\newline
Ohjaajat: Esa \textsc{Salmikangas} \& Marko \textsc{Rintam\"{a}ki}}

%------------------------------------------------

\vspace{1em} % Extra space between major sections
\vspace{1em} % Extra space between major sections

\vspace{2em} % Extra space between major sections
\NewEntry{2000--2005}{Jyv\"{a}skyl\"{a}n ammattikorkeakoulu}

\Description{\MarginText{Insin\"{o}\"{o}ri AMK}Tekniikka ja Liikenne\newline 
Lopputy\"{o}: \textit{Progress RDBMS/4GL - kehitys Eclipse-ymp\"{a}rist\"{o}ss\"{a}}
\newline
Kuvaus: Esiselvitys Progress-kehityksest\"{a} Eclipse-ymp\"{a}rist\"{o}ss\"{a}.
Erilaisten arkkitehtuuriratkaisujen vertailua ja analysointia.\newline
Ohjaaja: Jouni \textsc{Huotari}}

%------------------------------------------------

\vspace{1em} % Extra space between major sections

%------------------------------------------------------------------------------
%	PUBLICATIONS
%------------------------------------------------------------------------------

\spacedlowsmallcaps{Julkaisut}\vspace{1em}

\NewEntry{Marraskuu 2012}{Python, Behave and Mockito-Python}

\Description{\MarginText{Linux For You}Artikkeli BDD:n perusteista
Pythonilla ohjelmistokehitt\"{a}jille, jotka ovat kiinnostuneet automaatiosta.
Artikkelissa kuvataan BDD:n perusteet k\"{a}yt\"{a}nn\"{o}n n\"{a}k\"{o}kulmasta
samalla kun kirjoitetaan lyhyt esimerkkiohjelma.
\\ Kirjoittajat: Tuukka \textsc{Turto}}

\NewEntry{Syyskuu 2013}{Programming Can Be Fun with Hy}

\Description{\MarginText{Open Source For You}Artikkeli Hy-kielen alkeista. Hy
on Pythonilla kirjoitettu Lisp-toteutus, joka mahdollistaa saumattoman
integraation molempiin suuntiin. Kielen avulla voi kirjoittaa funktionaalista
koodia, samalla hy\"{o}dynt\"{a}en Pythonin laajaa moduulikirjastoa.
\\ Kirjoittajat: Tuukka \textsc{Turto}}

\NewEntry{Helmikuu 2014}{Top 5 Tools for Writing a Thesis}

\Description{\MarginText{Open Source For You}Viisi erilaista lopputy\"{o}n
kirjoittamiseen sopivaa ty\"{o}kalua esittelev\"{a} artikkeli.
\\ Kirjoittajat: Tuukka \textsc{Turto}}


%------------------------------------------------

%------------------------------------------------

\vspace{1em} % Extra space between major sections

%------------------------------------------------------------------------------
%	COMPUTER SKILLS
%------------------------------------------------------------------------------

\spacedlowsmallcaps{Taidot}\vspace{1em}

\Description{\MarginText{Perusteet}\textsc{Funktionaalinen ohjelmointi} (Hy,
Clojure), Web-ohjelmointi (Django, Dart)}

\Description{\MarginText{Jatko}\textsc{Jatkuva integraatio} (TeamCity, Travis CI), 
Hajautettu versionhallinta (Git), Keskitetty versionhallinta (SVN,
CM/Synergy), Relaatiotietokannat (SqlServer, Oracle, SQLite), K\"{a}ytt\"{o}liittym\"{a}t
(WinForms, PyQt, curses)}

\Description{\MarginText{Asiantuntija}\textsc{Olio-ohjelmointi} (VB.Net, Python,
C\#, Java), Testausautomaatio (NUnit, JUnit, nose, behave, PyHamcrest,
mockito-python, NSubstitute)}

% Funktionaalinen ohjelmointi (Hy, Clojure)

% Relaatiotietokannat (SqlServer, Oracle, SQLite)
% Hajautettu versionhallinta (Git)
% Keskitetty versionhallinta (SVN, CM/Synergy)
% Jatkuva integraatio (TeamCity, Travis CI)
% Kayttoliittymat (WinForms, PyQt, curses)

% Olio-ohjelmointi (VB.Net, Python, C#, Java)
% Testiautomaatio (NUnit, JUnit, nose, behave, PyHamcrest, mockito-python,
% NSubstitute)

%------------------------------------------------

\vspace{1em} % Extra space between major sections

%------------------------------------------------------------------------------
%	OTHER INFORMATION
%------------------------------------------------------------------------------

\spacedlowsmallcaps{Muut tiedot}\vspace{1em}


\Description{\MarginText{Muut projektit}2010--\ \ $\cdotp$\ \ \href{https://github.com/tuturto/pyherc}{pyherc},
Rogue-tyylinen luolastoseikkailu (Python 3, Hy ja PyQt)}

\vspace{-0.5em} % Negative vertical space to counteract the vertical space between every \Description command

\Description{2013--\ \ $\cdotp$\ \ \href{https://github.com/hylang/hy}{Hy}, Lisp-toteutus Pythonilla.}

%------------------------------------------------

\vspace{1em}

\Description{\MarginText{Yhdistystoiminta}2001--2002\ \ $\cdotp$\ \ Jyv\"{a}skyl\"{a}n insin\"{o}\"{o}rioppilaat ry}

\vspace{-0.5em} % Negative vertical space to counteract the vertical space between every \Description command

\Description{2001--2002\ \ $\cdotp$\ \ Jyv\"{a}skyl\"{a}n tekniikanopiskelijat ry}

%------------------------------------------------

%\vspace{1em}

%\Description{\MarginText{Communication Skills}2010\ \ $\cdotp$\ \ Oral Presentation at the California Business Conference}

%\vspace{-0.5em} % Negative vertical space to counteract the vertical space between every \Description command

%\Description{2009\ \ $\cdotp$\ \ Poster at the Annual Business Conference in Oregon}

%------------------------------------------------

\vspace{1em}

\newlength{\langbox} % Create a new length for the length of languages to keep them equally spaced
\settowidth{\langbox}{English} % Length equals the length of "English" - if you have a longer language in your list put it here

\Description{\MarginText{Kielitaito}\parbox{\langbox}{\textsc{suomi}}\ \ $\cdotp$\ \ \ \"{a}idinkieli}

\vspace{-0.5em} % Negative vertical space to counteract the vertical space between every \Description command

\Description{\parbox{\langbox}{\textsc{englanti}}\ \ $\cdotp$\ \ \ sujuva}

\vspace{-0.5em} % Negative vertical space to counteract the vertical space between every \Description command

\Description{\parbox{\langbox}{\textsc{ruotsi}}\ \ $\cdotp$\ \ \ alkeet}

\vspace{1em} % Negative vertical space to counteract the vertical space between every \Description command

%------------------------------------------------

\Description{\MarginText{Kiinnostukset}Ketter\"{a} ohjelmistokehitys\ 
\ $\cdotp$\ \ sosiaalinen koodaus\ \ $\cdotp$\ \ lukeminen\ \ $\cdotp$\ \ 
lautapelit \ }

%------------------------------------------------------------------------------

\end{cv}

\end{document}
